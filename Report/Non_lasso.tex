
\documentclass{article}% use option titlepage to get the title on a page of its own.

\linespread{1.4}

\usepackage{lmodern}
\usepackage{listings}
\usepackage{color}

\definecolor{dkgreen}{rgb}{0,0.6,0}
\definecolor{gray}{rgb}{0.5,0.5,0.5}
\definecolor{mauve}{rgb}{0.58,0,0.82}

\usepackage{graphicx}
\graphicspath{ {./images/} }



\title{%
  Non-negative LASSO \\
  \large A Portfolio replication application}

\date{2019, February}
\author{Giovanni Misseri, Andrea Della Vecchia \\ \\ 
Business Economics and Finantial Data}
\begin{document}
\maketitle
\tableofcontents
\newpage
\section{Introduction}

Portfolio management has always been one of the most used way investors start approaching to financial investments. An important task in finance is indeed optimizing the revenue with respect to the risk of investing in a portfolio.

Markowitz in the '50, stated that single minded pursuit of high returns results in a poor strategy and rational investors need to balance their desire for high returns with low risk. So investing in a portolio seems reasonable, indeed it diversify our investment and if we are able to spot profitable companies on which invest, we are guaranteed good revenue taking relatively low risk.
\\

There are mainly two stratgies in portfolio management, active and passive managements. Active management tries to exploit the fluctuations of the market to gain as much as possible. On the other hand, passive management, tries to mimic the performance of an index. As one can easily notice, with active approach it's possible to loose great amount of money even investing on, on average, growing companies. On the other hand if we are able to well mimic a, on average, growing index, with passive approach we are guaranteed positive revenue.

Empirically is often the case passive approach outperforms active approach, also due to the transaction costs. In this work will try to propose two different solution to pursue respectively the passive and the active approach.

\newpage
\section{Non negative LASSO}

\section{Portfolio replication}





\end{document}